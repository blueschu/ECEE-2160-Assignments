\documentclass[11pt, letterpaper]{article} % Copyright (c) 2020 Brian Schubert



\def\LABnumber{5}
\def\LABdatedue{August 5th, 2020}
\def\LABdatesubmitted{August 5th, 2020}
\def\LABtitle{Introduction to Quartus}


% ECEE 2160 Lab Report Header
%
% Copyright (c) 2020 Brian Schubert
% 
% This header file was created to format lab reports in my
% Embedded Design course taken Northeastern University during
% the Summer 2 2020 semester.
%
%

% Many packages are retained from previous lab report headers
% for future convenience. 
\usepackage{amsmath,amsfonts,amssymb,amsthm}
\usepackage[english]{babel}
\usepackage{caption}
\usepackage[makeroom]{cancel}
\usepackage[dvipsnames,table]{xcolor}
\usepackage{enumitem}
\usepackage{esint}
\usepackage{geometry}
\usepackage{graphicx}
\usepackage{hyperref}
\usepackage[utf8]{inputenc}
\usepackage{listings} % For code listings
% Replace times with this package to typeset math in times aswell
% Note: no bold typeface exists for the times font, so \boldsymbol
% does not work when using a times font in math mode.
%\usepackage{mathptmx}
\usepackage{mathtools}
\usepackage{multirow}
\usepackage{pgfplots}
\usepackage{placeins} % For FloatBarrier
\usepackage{siunitx}
\usepackage{subcaption}
\usepackage{tikz}
\usepackage[american]{circuitikz}
\usepackage{todonotes}
\usepackage{times} % For times font
\usepackage[explicit]{titlesec}

% Use deja-vu sans mono for monospaced typeface
% Must be after times font package.
\usepackage[scaled=0.9]{DejaVuSansMono}

\geometry{left=1in,right=1in,top=1in,bottom=1in}

% File prefixes to use when searching for graphics.
\graphicspath{ {../images/} {./images/} }

% Color all used links blue.
\hypersetup{
    colorlinks=true,
    linkcolor=MidnightBlue,   
    citecolor=MidnightBlue,   
    urlcolor=MidnightBlue,
}

\urlstyle{same} % Print URLs using the surroudning font instead of forcing monospace

% Macro to generate the an IPL template style title page
\newcommand{\makelabtitle}{
\begin{titlepage}
    \begin{center}
    \renewcommand{\baselinestretch}{1.5}
    
    \includegraphics[width=\textwidth]{northeastern-header}\par
    
    \vspace{0.4in}

    {\Large
        \LABcourse\par
        \LABcoursetitle\par
        \LABsemester\par
    }

    \vspace{0.4in}

    {\LARGE\bfseries
        Report for Lab Assignment \LABnumber\par
        \ifx \LABtitle \undefined
        \else
            \LABtitle\par
        \fi
    }
    
    \vspace{1.0in}
    
    {\Large \LABauthor\par
        \ifx \LABpartner \undefined
        \else
            \textbf{Lab Partner}: \LABpartner\\
        \fi
        \textbf{Instructor:} \LABinstructor\par
        \LABdate\par
     }
    \end{center}
    \vfill\null
    \section*{Abstract}
    \LABabstract
\end{titlepage}
}

\titleformat{\section}{\normalfont\bfseries}{\thesection}{1em}{\underline{#1}}

% Notation Definitions
\newcommand{\dd}[1]{\mathrm{d}#1}
\newcommand{\diffop}[2][]{\frac{\dd#1}{\dd #2}}
\newcommand{\pdiffop}[2][]{\frac{\partial #1}{\partial #2}}
\newcommand{\bvec}[1]{{\vec{\boldsymbol{#1}}}}
\newcommand{\relerr}[1]{\frac{\delta #1}{#1}}
\newcommand{\unitvec}[1]{\hat{\boldsymbol{#1}}}

\sisetup{inter-unit-product =\cdot}
\sisetup{per-mode = symbol}
\sisetup{separate-uncertainty = true}
\sisetup{multi-part-units=single}
\DeclareSIUnit{\radian}{rad}
\DeclareSIUnit{\degreeFahrenheit}{{}^\circ F}

% Use computer modern roman fontface for emf symbol.
% https://tex.stackexchange.com/questions/67881/resetting-mathcal-font-to-default
\DeclareMathAlphabet{\defaultmathcal}{OMS}{cmsy}{m}{n}
\newcommand{\emf}{\defaultmathcal{E}}



% Print table and figure labels in bold font
\captionsetup[table]{labelfont=bf}
\captionsetup[figure]{labelfont=bf}

%\captionsetup[lstlisting]{ format=listing, labelfont=white, textfont=white, singlelinecheck=false, margin=0pt, font={bf,footnotesize} }
\captionsetup[lstlisting]{labelfont=bf}

% Make table columns slighly wider by default
\setlength{\tabcolsep}{8pt}

% Custom code color scheme
\definecolor{codekeyword}{RGB}{3,0,130}
\definecolor{codeidentifier}{RGB}{20, 20, 80}
\definecolor{codestring}{RGB}{72, 140, 2}
\definecolor{codecomment}{rgb}{0.3,0.3,0.3}
\definecolor{codedirective}{RGB}{160, 130, 60}


% Custom listings styling
\lstdefinestyle{labreportstyle}{
    % Syntax highlighting
    basicstyle=\ttfamily\footnotesize,
    backgroundcolor=\color{white},
    commentstyle=\itshape\color{codecomment},
    keywordstyle=\bfseries\color{codekeyword},
    stringstyle=\bfseries\color{codestring},
    identifierstyle=\color{codeidentifier},
    % Tab width and display
    tabsize=4,
    showtabs=true,
    % Don't show spaces in strings
    showspaces=false,
    showstringspaces=false,
    % Show line numbers
    numbers=left,
    numberstyle=\scriptsize\sffamily\color{gray},
    % Allow listing to break long lines
    breaklines=true,
    breakatwhitespace=true,
    % Frame settings
    frame=lines,
    % Misc
    captionpos=t,
    keepspaces=true,
    %    numbersep=5pt,
}

% C++ specific styling
\lstdefinestyle{labreportstyle-C++}{
    language=C++,	% Not sure if redundant now, but was required at one point
    style=labreportstyle,
    directivestyle={\color{codedirective}},
    morekeywords={constexpr,nullptr}
}

% shell specific styling
\lstdefinestyle{labreportstyle-sh}{
    language=sh,
    style=labreportstyle,
    numbers=none,
    identifierstyle=\color{black},
    emph={\$},
    emphstyle=\bfseries\color{codekeyword},
}

% Conveience macro to load code file listings.
%
% Note that the style is automatically selected based on the language paramater.
\newcommand{\includecode}[2][C++]{%
    \lstinputlisting[caption={\texttt{#2}}, label={lst:#2}, language=#1, style=labreportstyle-#1]{#2}
}

\def\LABreportheading{Prelab Assignment for Lab}
\rhead{\LABcoursetitle\\Prelab for Lab Assignment \LABnumber}



\begin{document}
\makelabtitle
  
\section*{Summary}

This prelab assignment consisted of preparing truth tables and circuit diagrams for the half- and full-adder.

\section*{Submission}

\begin{table}[h]\centering
    \caption{Prelab Truth Tables.}
    \def\arraystretch{1.2}
    \begin{subtable}[t]{0.48\linewidth}\centering
        \caption{Half-Adder}
        \begin{tabular}{|cc|cc|}
            \hline
            \multicolumn{2}{|m{1.5cm}|}{\bfseries\centering Inputs} & 
            \multicolumn{2}{m{1.5cm}|}{\bfseries\centering Outputs}\\
            $\boldsymbol{A}$ & $\boldsymbol{B}$ & $\boldsymbol{C}$ & $\boldsymbol{S}$\\
            \hline
            0 & 0 & 0 & 0\\
            0 & 1 & 0 & 1\\
            1 & 0 & 0 & 1\\
            1 & 1 & 1 & 0\\
            \hline
        \end{tabular}
    \end{subtable}
    \hfill\null
    \begin{subtable}[t]{0.48\linewidth}\centering
        \caption{Full-Adder}
        \begin{tabular}{|ccc|cc|}
            \hline
            \multicolumn{3}{|m{2.25cm}|}{\bfseries\centering Inputs} & 
            \multicolumn{2}{m{1.5cm}|}{\bfseries\centering Outputs}\\
            $A$ & $B$ &  $C_\mathrm{in}$ & $C_\mathrm{out}$ & $S$\\
            \hline
            0 & 0 & 0 & 0 & 0\\
            0 & 0 & 1 & 0 & 1\\
            0 & 1 & 0 & 0 & 1\\
            0 & 1 & 1 & 1 & 0\\
            1 & 0 & 0 & 0 & 1\\
            1 & 0 & 1 & 1 & 0\\
            1 & 1 & 0 & 1 & 0\\
            1 & 1 & 1 & 1 & 1\\
            \hline
        \end{tabular}
    \end{subtable}
\end{table}

In preparation for creating the circuit diagrams, these truth tables were translated into boolean equations and then reduced using boolean algebra identities as follows \cite{fell-discrete-structure}.

\paragraph{Half-Adder}
\begin{align*}
    S(A,B) &= A'B + AB' = A\oplus B \\
    C(A,B) &= A B
\end{align*}

\paragraph{Full-Adder}
\begin{align*}
    S(A,B,C_\mathrm{in}) &= A'B'C_\mathrm{in} + A'BC_\mathrm{in}' + AB'C_\mathrm{in}' + ABC_\mathrm{in}\\
        &= A'(B'C_\mathrm{in} + BC_\mathrm{in}') + A(B'C_\mathrm{in}' + BC_\mathrm{in})\\
        &= A'\left(B \oplus C\right) +A((B + C_\mathrm{in})' +BC_\mathrm{in})\\
         &= A'\left(B \oplus C\right) +A(B\oplus C_\mathrm{in})'\\
        &= A \oplus (B \oplus C_\mathrm{in})\\
    C_\mathrm{out}(A,B,C_\mathrm{in}) &= A'BC_\mathrm{in} + AB'C_\mathrm{in} + ABC_\mathrm{in}' + ABC_\mathrm{in}\\
        &= C_\mathrm{in} (A'B + AB') + AB\cancelto{1}{(C_\mathrm{in}' + C_\mathrm{in})}\\
        &= C_\mathrm{in}\left(A\oplus B\right) + AB
\end{align*}

These equation were then translated to the following circuit diagrams.

\begin{figure}[h]\centering
    \begin{subfigure}{0.38\linewidth}\centering
        \begin{circuitikz}
            \node[xor port] (xor1) at (0, 0) {};
            \node[and port] (and1) at (0,-2) {};
            
            \draw (xor1.in 1) to[short] ++ (-1, 0) node[left] {$A$};
            \draw (xor1.in 2) to[short] ++ (-1, 0) node[left] {$B$};
            \draw (xor1.out) to[short] ++ (1, 0) node[right] {$S$};
            
            \draw (and1.in 1) to[short] ++ (-0.2, 0) node (n1) {} to[short, -*] (n1 |- xor1.in 1);
            \draw (and1.in 2) to[short] ++ (-0.6, 0) node (n2) {} to[short, -*] (n2 |- xor1.in 2);
            \draw (and1.out) to[short] ++ (1, 0) node[right] {$C$};
            
        \end{circuitikz}
        \caption{Half-Adder}
    \end{subfigure}
    \begin{subfigure}{0.58\linewidth}\centering
        \begin{circuitikz}
            \node[xor port] (xor1) at (0, 0) {};    % A xor B
            \node[and port] (and1) at (0,-3) {};    % A and B
            \node[and port] (and2) at (2,-2) {};    % (A xor B) and C_in
            \node[xor port] (xor2) at (3,-0.5) {};  % (A xor B) xor C_in
            \node[or port] (or1) at (4,-2.75) {};   % (A and B) or ((A xor B) and C_in)
            
            % A xor B
            \draw (xor1.in 1) to[short] ++ (-1, 0) node[left] {$A$};
            \draw (xor1.in 2) to[short] ++ (-1, 0) node[left] {$B$};
            
            % A and B
            \draw (and1.in 1) to[short] ++ (-0.2, 0) node (n1) {} 
                to[short, -*] (n1 |- xor1.in 1);
            \draw (and1.in 2) to[short] ++ (-0.6, 0) node (n2) {} 
                to[short, -*] (n2 |- xor1.in 2);
            
            % (A xor B) xor C_in
            \draw (xor2.in 1) to[short] ++ (-0.5, 0) node (n3) {}
                to[short] (n3 |- xor1.out) 
                to[short] (xor1.out);
            \draw (xor2.in 2) to[short] ++ (-4, 0) node[left] (cin) {$C_\mathrm{in}$};
            \draw (xor2.out) to[short] ++ (1.25, 0) node[right] {$S$};
            
            % (A xor B) and C_in
            \draw (and2.in 1) to[short] ++ (-0.25, 0) node (n4) {}
                to[short, -*] (n4 |- xor1.out);
            \draw (and2.in 2) to[short] ++ (-0.75, 0) node (n5) {}
                to[short, -*] (n5 |- cin);
            
            % (A and B) or ((A xor B) and C_in)
            \draw (or1.in 1) to[short] (or1.in 1 |- and2.out)
                to[short] (and2.out);
            \draw (or1.in 2) to[short] (and1.out);
             \draw (or1.out) to[short] ++ (0.25, 0) node[right] {$C_\mathrm{out}$};
                
        \end{circuitikz}
        \caption{Full-Adder}
    \end{subfigure}
    \caption{Circuit diagrams for half- and full-adder.}
\end{figure}

% Post lab: style for drawing logic circuits:
% https://tex.stackexchange.com/questions/449025/how-to-draw-logic-gates-in-tikz

\bibliography{../ecee-2160-common.bib,./ecee-2160-lab-5.bib}

\bibliographystyle{unsrt}


\end{document}