\documentclass[11pt, letterpaper]{article} % Copyright (c) 2020 Brian Schubert



\def\LABnumber{2}
\def\LABdatedue{July 15th, 2020}
\def\LABdatesubmitted{July 15th, 2020}
\def\LABtitle{Linked Lists and the \texttt{gdb} Debugger}


% ECEE 2160 Lab Report Header
%
% Copyright (c) 2020 Brian Schubert
% 
% This header file was created to format lab reports in my
% Embedded Design course taken Northeastern University during
% the Summer 2 2020 semester.
%
%

% Many packages are retained from previous lab report headers
% for future convenience. 
\usepackage{amsmath,amsfonts,amssymb,amsthm}
\usepackage[english]{babel}
\usepackage{caption}
\usepackage[makeroom]{cancel}
\usepackage[dvipsnames,table]{xcolor}
\usepackage{enumitem}
\usepackage{esint}
\usepackage{geometry}
\usepackage{graphicx}
\usepackage{hyperref}
\usepackage[utf8]{inputenc}
\usepackage{listings} % For code listings
% Replace times with this package to typeset math in times aswell
% Note: no bold typeface exists for the times font, so \boldsymbol
% does not work when using a times font in math mode.
%\usepackage{mathptmx}
\usepackage{mathtools}
\usepackage{multirow}
\usepackage{pgfplots}
\usepackage{placeins} % For FloatBarrier
\usepackage{siunitx}
\usepackage{subcaption}
\usepackage{tikz}
\usepackage[american]{circuitikz}
\usepackage{todonotes}
\usepackage{times} % For times font
\usepackage[explicit]{titlesec}

% Use deja-vu sans mono for monospaced typeface
% Must be after times font package.
\usepackage[scaled=0.9]{DejaVuSansMono}

\geometry{left=1in,right=1in,top=1in,bottom=1in}

% File prefixes to use when searching for graphics.
\graphicspath{ {../images/} {./images/} }

% Color all used links blue.
\hypersetup{
    colorlinks=true,
    linkcolor=MidnightBlue,   
    citecolor=MidnightBlue,   
    urlcolor=MidnightBlue,
}

\urlstyle{same} % Print URLs using the surroudning font instead of forcing monospace

% Macro to generate the an IPL template style title page
\newcommand{\makelabtitle}{
\begin{titlepage}
    \begin{center}
    \renewcommand{\baselinestretch}{1.5}
    
    \includegraphics[width=\textwidth]{northeastern-header}\par
    
    \vspace{0.4in}

    {\Large
        \LABcourse\par
        \LABcoursetitle\par
        \LABsemester\par
    }

    \vspace{0.4in}

    {\LARGE\bfseries
        Report for Lab Assignment \LABnumber\par
        \ifx \LABtitle \undefined
        \else
            \LABtitle\par
        \fi
    }
    
    \vspace{1.0in}
    
    {\Large \LABauthor\par
        \ifx \LABpartner \undefined
        \else
            \textbf{Lab Partner}: \LABpartner\\
        \fi
        \textbf{Instructor:} \LABinstructor\par
        \LABdate\par
     }
    \end{center}
    \vfill\null
    \section*{Abstract}
    \LABabstract
\end{titlepage}
}

\titleformat{\section}{\normalfont\bfseries}{\thesection}{1em}{\underline{#1}}

% Notation Definitions
\newcommand{\dd}[1]{\mathrm{d}#1}
\newcommand{\diffop}[2][]{\frac{\dd#1}{\dd #2}}
\newcommand{\pdiffop}[2][]{\frac{\partial #1}{\partial #2}}
\newcommand{\bvec}[1]{{\vec{\boldsymbol{#1}}}}
\newcommand{\relerr}[1]{\frac{\delta #1}{#1}}
\newcommand{\unitvec}[1]{\hat{\boldsymbol{#1}}}

\sisetup{inter-unit-product =\cdot}
\sisetup{per-mode = symbol}
\sisetup{separate-uncertainty = true}
\sisetup{multi-part-units=single}
\DeclareSIUnit{\radian}{rad}
\DeclareSIUnit{\degreeFahrenheit}{{}^\circ F}

% Use computer modern roman fontface for emf symbol.
% https://tex.stackexchange.com/questions/67881/resetting-mathcal-font-to-default
\DeclareMathAlphabet{\defaultmathcal}{OMS}{cmsy}{m}{n}
\newcommand{\emf}{\defaultmathcal{E}}



% Print table and figure labels in bold font
\captionsetup[table]{labelfont=bf}
\captionsetup[figure]{labelfont=bf}

%\captionsetup[lstlisting]{ format=listing, labelfont=white, textfont=white, singlelinecheck=false, margin=0pt, font={bf,footnotesize} }
\captionsetup[lstlisting]{labelfont=bf}

% Make table columns slighly wider by default
\setlength{\tabcolsep}{8pt}

% Custom code color scheme
\definecolor{codekeyword}{RGB}{3,0,130}
\definecolor{codeidentifier}{RGB}{20, 20, 80}
\definecolor{codestring}{RGB}{72, 140, 2}
\definecolor{codecomment}{rgb}{0.3,0.3,0.3}
\definecolor{codedirective}{RGB}{160, 130, 60}


% Custom listings styling
\lstdefinestyle{labreportstyle}{
    % Syntax highlighting
    basicstyle=\ttfamily\footnotesize,
    backgroundcolor=\color{white},
    commentstyle=\itshape\color{codecomment},
    keywordstyle=\bfseries\color{codekeyword},
    stringstyle=\bfseries\color{codestring},
    identifierstyle=\color{codeidentifier},
    % Tab width and display
    tabsize=4,
    showtabs=true,
    % Don't show spaces in strings
    showspaces=false,
    showstringspaces=false,
    % Show line numbers
    numbers=left,
    numberstyle=\scriptsize\sffamily\color{gray},
    % Allow listing to break long lines
    breaklines=true,
    breakatwhitespace=true,
    % Frame settings
    frame=lines,
    % Misc
    captionpos=t,
    keepspaces=true,
    %    numbersep=5pt,
}

% C++ specific styling
\lstdefinestyle{labreportstyle-C++}{
    language=C++,	% Not sure if redundant now, but was required at one point
    style=labreportstyle,
    directivestyle={\color{codedirective}},
    morekeywords={constexpr,nullptr}
}

% shell specific styling
\lstdefinestyle{labreportstyle-sh}{
    language=sh,
    style=labreportstyle,
    numbers=none,
    identifierstyle=\color{black},
    emph={\$},
    emphstyle=\bfseries\color{codekeyword},
}

% Conveience macro to load code file listings.
%
% Note that the style is automatically selected based on the language paramater.
\newcommand{\includecode}[2][C++]{%
    \lstinputlisting[caption={\texttt{#2}}, label={lst:#2}, language=#1, style=labreportstyle-#1]{#2}
}

\def\LABreportheading{Prelab Assignment for Lab}
\rhead{\LABcoursetitle\\Prelab for Lab Assignment \LABnumber}



\begin{document}
  \makelabtitle
  
\section*{Summary}

This prelab assignment consisted of two parts. In the first part, the \texttt{gdb} debugger was used to introspect the execution of an instructor provided C++ program. In the second part, an interactive menu program was prepared, which will be used for testing a linked list implementation during the lab. Both parts were completed on a host machine running a GNU/Linux operating system\footnote{Linux Mint 19.3, ker. 4.15.0, arch. x86\_64.}. Programs were compiled using the GNU Compiler Collection (GCC) C++ compiler version 7.5.0., and debugged using the GNU \texttt{gdb} debugger version 8.1.

\section*{Submission}

\subsection*{Part I}
The instructor-provided \texttt{person.cpp} file may be found in Listing~\ref{lst:instructor/person.cpp} below. This file was compiled into an executable with debug info using the following shell command.
\begin{lstlisting}[style=labreportstyle-sh,escapechar=!]
$ g++ -std=c++17 -Wall -Wextra -o person ./person.cpp       # Compile executable
$ file ./person                                             # Confirm added debug info
person: ELF 64-bit LSB shared object, x86-64, version 1 (SYSV), dynamically linked, interpreter /lib64/ld-linux-x86-64.so.2, for GNU/Linux 3.2.0, BuildID[sha1]=cd8698296f4f24be431e5dd622c4e2d09bae7707, !\bfseries\colorbox{yellow!30!orange!20!white}{with debug\_info}!, not stripped
\end{lstlisting}
The execution of this program was then analyzed with the following \texttt{gdb} session. Comments and explanations are included on the right hand side following a \texttt{\#} sign.
\begin{lstlisting}[style=labreportstyle-sh,escapechar=!]
$ gdb
(gdb) file person                                       # Load the object file person
Reading symbols from person...done.
(gdb) start                                             # Begin execution
Temporary breakpoint 1 at 0x1a5e: file ./person.cpp, line 32.
Starting program: /path/to/person 

Temporary breakpoint 1, main () at ./person.cpp:32
32      {
(gdb) next                                              # Advance to next source line
33          Person person;
(gdb) next                                              # Advance to next source line
34          person.name = "John";
(gdb) p person                          # The author of the program chose not to default
$1 = {name = "", age = 1431660720}      # initalize person, so the contents of person are
                                        # arbitrary. Reading in the program would result in UB.
(gdb) next                                              # Advance to next source line
35          person.age = 10;
(gdb) next                                              # Advance to next source line
36          PrintPerson(&person);
(gdb) p person                                              # Print contents of person
$2 = {name = "John", age = 10}
(gdb) p &person                                             # Print address of person
$3 = ((anonymous namespace)::Person *) 0x7fffffffd250       # N.B. this address is on the stack
(gdb) p sizeof(person)
$4 = 40                                                     # Size of person in CHAR_BIT bytes
(gdb) print person.name.c_str()         # Address of the storage used by person.name
$5 = 0x7fffffffd260 "John"              # Notably, this address is on the stack, not the heap
(gdb) p &person.age
$6 = (int *) 0x7fffffffd270             # Address of age within person structure
(gdb) x/40bx &person                    # Examine the contents of person (40 bytes, in hex)
0x7fffffffd250: 0x60    0xd2    0xff    0xff    0xff    0x7f    0x00    0x00
0x7fffffffd258: 0x04    0x00    0x00    0x00    0x00    0x00    0x00    0x00  # Highlighted:
0x7fffffffd260:!\bfseries\colorbox{yellow!30!orange!20!white}{0x4a\ \ \ \ 0x6f\ \ \ \ 0x68\ \ \ \ 0x6e\ \ \ \ 0x00}!    0x7f    0x00    0x00    # J O H N \0
0x7fffffffd268: 0x00    0x00    0x00    0x00    0x00    0x00    0x00    0x00
0x7fffffffd270:!\bfseries\colorbox{yellow!30!orange!20!white}{0x0a}!    0x00    0x00    0x00    0x55    0x55    0x00    0x00    # 0x0a == 10
(gdb) x/5bc 0x7fffffffd260
0x7fffffffd260: 74 'J'  111 'o' 104 'h' 110 'n' 0 '\000'
# This output shows that the string "John" is stored within the std::string component of
# person rather than on the heap. This is due to an optimization for short strings
# implemented in the GNU C++ standard library.
(gdb) step                                      # Advance program; enter function call
(anonymous namespace)::PrintPerson (person=0x7fffffffd250) at ./person.cpp:26
26          cout << person->name << " is " << person->age << " years old\n";
(gdb) p person
$7 = ((anonymous namespace)::Person *) 0x7fffffffd250     # Print the passed pointer; matches
                                                          # the address of person in main
(gdb) next                                              # Advance to next source line
John is 10 years old
27      }
(gdb) next                                              # Advance to next source line
main () at ./person.cpp:33
33          Person person;
(gdb) next                                              # Advance to next source line
37      }
(gdb) next                                              # Advance to next source line
[Inferior 1 (process 22465) exited normally]
\end{lstlisting}


\subsection*{Part II}
The part consisted of preparing an interactive menu program that could be used to test a linked list implementation during the lab. This program is provided in Listing~\ref{lst:prelab.cpp} in the Appendix. 

The prelab program was compiled and tested with the following shell session. User inputs were provided as requested by the program.

%\begin{multicols}{2}
\begin{lstlisting}[style=labreportstyle-sh]
$ g++ -std=c++17 -Wall -Wextra \
>    -o lab2-prelab ./prelab.cpp
$ ./lab2-prelab
====== Menu ======
1. Add a person
2. Find a person
3. Remove a person
4. Print the list
5. Exit

Select an option: 1
You selected "Add a person"

====== Menu ======
1. Add a person
2. Find a person
3. Remove a person
4. Print the list
5. Exit

Select an option: 2
You selected "Find a person"

====== Menu ======
1. Add a person
2. Find a person
3. Remove a person
4. Print the list
5. Exit

Select an option: 3
You selected "Remove a person"

====== Menu ======
1. Add a person
2. Find a person
3. Remove a person
4. Print the list
5. Exit

Select an option: 4
You selected "Print the list"

====== Menu ======
1. Add a person
2. Find a person
3. Remove a person
4. Print the list
5. Exit

Select an option: A
Invalid input
Select an option: 20
Invalid selection - selection must be an integer from [1,5]

====== Menu ======
1. Add a person
2. Find a person
3. Remove a person
4. Print the list
5. Exit

Select an option: 5
Exiting...
\end{lstlisting}
%\end{multicols}


\clearpage
\section*{Appendix}
\renewcommand{\thelstlisting}{A.\arabic{lstlisting}}

\includecode{instructor/person.cpp}

\includecode{prelab.cpp}


\end{document}